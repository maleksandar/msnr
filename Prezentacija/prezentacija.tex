\documentclass[10pt]{beamer}

\usetheme[numbering=fraction,progressbar=frametitle]{metropolis}
\usepackage{appendixnumberbeamer}

\usepackage{booktabs}
\usepackage[scale=2]{ccicons}
\usepackage[utf8]{inputenc}
\usepackage[english,serbian]{babel}
\usepackage[T2A]{fontenc} % enable Cyrillic fonts
\usepackage{pgfplots}
\usepgfplotslibrary{dateplot}
\usepackage{listings}


\usepackage{xspace}
\newcommand{\themename}{\textbf{\textsc{metropolis}}\xspace}

\title{ Optimizacija izvršavanja programa kroz Iterativnu Kompilaciju }
\date{\today}
\date{}
\author{Aleksandar Preočanin, Filip Novović, Aleksandar Milosavljević}
\institute{Univerzitet u Beogradu, Matematički fakultet}

\begin{document}

\maketitle

%%%% ŠTA? %%%%
\begin{frame}[standout]
  Šta je iterativna kompilacija?
\end{frame}

\begin{frame}[fragile]{Šta je iterativna kompilacija?}

  {\textbf{\textsc{Iterativna kompilacija}}\xspace} je metoda pronalaženja ulaznih parametara statičkih transformacija
  izvornog koda za koje se dobija kod sa najboljim performansama pri izvršavanju prevedenog programa.
\end{frame}

\begin{frame}[fragile]{Šta je iterativna kompilacija?}
  \begin{itemize}[<+- | alert@+>]
    \item \textbf{Metoda} koja koristi standardni kompajler
    \item \textbf{Nije} posebna vrsta kompajlera
    \item Koristi tehnike \textbf{transformacije koda}
    \item \textbf{Traži} idealne ulazne parametre tih transformacija
    \item Pretragu vrši \textbf{iterativnom} metodom
  \end{itemize}
\end{frame}

\begin{frame}[standout]
  Zašto koristiti iterativnu kompilaciju?
\end{frame}
\begin{frame}[fragile]{Zašto koristiti iterativnu kompilaciju?}
  Iterativna kompilacija je ključna metoda u sistemima u kojima su performanse od suštinskog značaja.
  \begin{itemize}[<+- | alert@+>]
    \item Mreža pametnih uređaja (IoT)
    \item Uređaji bez pristupa konstantnom napajanju
    \item {...}
  \end{itemize}
\end{frame}


\begin{frame}[fragile]{Zašto koristiti iterativnu kompilaciju?}
  Kompajliranje iterativnom metodom ima za cilj da ubrza izvršavanje prevedenog programa ili dela programa.
\end{frame}

\begin{frame}[fragile]{Zašto koristiti iterativnu kompilaciju?}
  Ponekad optimizacije koje se mogu primeniti na kod nisu očigledne.
\end{frame}

\begin{frame}[fragile]{Zašto koristiti iterativnu kompilaciju?}
  Vrlo često se dešava da primena dve različite optimizacije koje pojedinačno imaju pozitivan efekat, kada se primene zajedno daju rezultat koji je sporiji od verzije bez optimizacije.
\end{frame}

\begin{frame}[fragile]{Zašto koristiti iterativnu kompilaciju?}
  Suština metode iterativne kompilacije je upravo u \textbf{pronalaženju} najbolje kombinacije tih transformacija.
  
  \vspace{5mm} %5mm vertical space  
  
  Pretraga tih kombinacija i njihova evaluacija vrši se iterativnom metodom.
\end{frame}

%%%% KAKO? %%%%
\begin{frame}[standout]
  Kako implementirati metod iterativne kompilacije?
\end{frame}
\begin{frame}[fragile]{Kako implementirati metod iterativne kompilacije?}
  Optimizacija koda u kontekstu iterativne kompilacije se vrši iz dva koraka.
  \begin{itemize}[<+- | alert@+>]
    \item Primenom \textbf{transformacija} nad kodom utičemo na resurse potrebne za njegovo izvršavanje
    \item Kombinaciju transformacija(zajedno sa parametrima) koja daje optimalan kod nalazimo uz pomoć \textbf{algoritma pretrage}
  \end{itemize}
\end{frame}

\begin{frame}[standout]
  Šta su metode transformacije?
\end{frame}
\begin{frame}[fragile]{Šta su metode transformacije?}

  {\textbf{\textsc{Transformacije}}\xspace}  su izmene koje se vrše nad kodom kako bi se dobile željene performanse pri njegovom izvršavanju uz uslov da izmenjen kod mora da bude semantički ekvivalentan originalnom.
\end{frame}

\begin{frame}[fragile]{Odmotavanje petlji (eng. loop unrolling)}
\begin{lstlisting}

 int x;
 for (x = 0; x < 100; x++)
 {
     delete(x);
 }
\end{lstlisting}
\end{frame}

\begin{frame}[fragile]{Odmotavanje petlji (eng. loop unrolling)}
\begin{lstlisting}
 int x; 
 for (x = 0; x < 100; x += 5)
 {
     delete(x);
     delete(x + 1);
     delete(x + 2);
     delete(x + 3);
     delete(x + 4);
 }
\end{lstlisting}
\end{frame}

\begin{frame}[fragile]{Podela na blokove (eng. loop tiling)}
\begin{lstlisting}
  int i, j, a[100][100], b[100], c[100];
  int n = 100;
  for (i = 0; i < n; i++) {
    c[i] = 0;
    for (j = 0; j < n; j++) {
      c[i] = c[i] + a[i][j] * b[j];
    }
  }
\end{lstlisting}
\end{frame}

\begin{frame}[fragile]{Podela na blokove (eng. loop tiling)}
\begin{lstlisting}
  int i, j, x, y, a[100][100], b[100], c[100];
  int n = 100;
  for (i = 0; i < n; i += 2) {
    c[i] = 0;
    c[i + 1] = 0;
    for (j = 0; j < n; j += 2) {
      for (x = i; x < min(i + 2, n); x++) {
        for (y = j; y < min(j + 2, n); y++) {
          c[x] = c[x] + a[x][y] * b[y];
        }
      }
    }
  }
\end{lstlisting}
\end{frame}

\begin{frame}[fragile]{Podela petlji (eng. loop fission)}
\begin{lstlisting}
  int i, a[100], b[100];
  for (i = 0; i < 100; i++)
  {
    a[i] = 1; 
    b[i] = 2;
  }
\end{lstlisting}
\end{frame}

\begin{frame}[fragile]{Podela petlji (eng. loop fission)}
\begin{lstlisting}
  int i, a[100], b[100];
  for (i = 0; i < 100; i++)
    a[i] = 1;                     
  for (i = 0; i < 100; i++)
    b[i] = 2;
\end{lstlisting}
\textbf{Spajanje petlji (eng. loop fusion)} predstavlja obrnutu transformaciju.
\end{frame}

\begin{frame}[standout]
  Šta su algoritmi pretrage?
\end{frame}
\begin{frame}[fragile]{Šta su algoritmi pretrage?}

\textbf{Prostor pretrage} (eng. search space) sadrži različite kombinacije parametara za koje se testira efikasnost kompiliranog 
programa. Kako bismo efikasno tražili optimalnu kombinaciju parametara koju ćemo na kraju i koristiti potrebno je da se što 
lakše krećemo po prostoru pretrage, zato koristimo \textbf{algoritme pretrage}.
\end{frame}

\begin{frame}[fragile]{Šta su algoritmi pretrage?}
Neki algoritmi pretrage koji se koriste u iterativnoj kompilaciji:
\begin{itemize}[<+- | alert@+>]
    \item \textbf{Genetski algoritmi}
    \item \textbf{Pretraga mreže}
    \item \textbf{Slučajna pretraga}

  \end{itemize}
\end{frame}

\begin{frame}[standout]
  Kako oceniti dobijeno rešenje?
\end{frame}

\begin{frame}{Kako oceniti dobijeno rešenje?}
  
  Nakon svake iteracije potrebno je testirati i oceniti dobijeno rešenje.

\end{frame}

\begin{frame}{Testiranje}

  \begin{itemize}[<+- | alert@+>]
    \item Testiranje se vrši na određenom skupu repera
    \item \textbf{Reper} predstavljaja ulaz programa koji se testira
    \item Kada imamo više od jednog repera dobijeni program treba izvršiti na svakom od njih
  \end{itemize}

\end{frame}

\begin{frame}{Ocenjivanje}

   Šta ćemo uzeti kao ocenu rešenja isključivo zavisi od toga šta želimo da postignemo.

\end{frame}

\begin{frame}{Ocenjivanje}

    Najčešće korišćene ocene su:
    \begin{itemize} 
      \item Vreme izvršavanja
      \item Potrošnja energije
      \item Kombinacija prethodna dva
    \end{itemize}

\end{frame}

% \section{Zaključak}
\begin{frame}[standout]
  Zaključak
\end{frame}

\begin{frame}{Zaključak}

  Rezultati koji su prikazani u seminarskom radu su
  pokazali da se iterativnom kompilacijom dolazi do 
  značajnih poboljšanja kako u vremenu izvršavanja tako i 
  u uštedi energije.

\end{frame}

\begin{frame}{Zaključak}

  \begin{itemize}[<+- | @alert->]
    \item Iterativna kompilacija daje bolje rezultate od statičke kompilacije
    \item Pogodno u embeded razvoju gde su brzina i ušteda energije od velikog značaja
    \item Mane ovog pristupa su što se vreme kompilacije znatno povećava
  \end{itemize}

\end{frame}

\begin{frame}[standout]
  Pitanja?
\end{frame}
\begin{frame}[standout]
  Hvala na pažnji!
\end{frame}

\end{document}
