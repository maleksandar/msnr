% !TEX encoding = UTF-8 Unicode

\documentclass[a4paper]{article}

\usepackage{color}
\usepackage{url}
\usepackage[T2A]{fontenc} % enable Cyrillic fonts
\usepackage[utf8]{inputenc} % make weird characters work
\usepackage{graphicx}

\usepackage[english,serbian]{babel}
%\usepackage[english,serbianc]{babel} %ukljuciti babel sa ovim opcijama, umesto gornjim, ukoliko se koristi cirilica

\usepackage[unicode]{hyperref}
\hypersetup{colorlinks,citecolor=green,filecolor=green,linkcolor=blue,urlcolor=blue}

%\newtheorem{primer}{Пример}[section] %ćirilični primer
\newtheorem{primer}{Primer}[section]

\begin{document}

\title{Optimizacija programa kroz različite pristupe Iterativnoj Kompilaciji\\ \small{Seminarski rad u okviru kursa\\Metodologija stručnog i naučnog rada\\ Matematički fakultet}}

\author{Prvi autor, drugi autor (treći autor)\\ kontakt email prvog, drugog (trećeg) autora}
\date{9.~april 2015.}
\maketitle

\abstract{
U ovom tekstu je ukratko prikazana osnovna forma seminarskog rada. Obratite pažnju da je pored ove .pdf datoteke, u prilogu i odgovarajuća .tex datoteka, kao i .bib datoteka korišćena za generisanje literature. Na prvoj strani seminarskog rada su naslov, apstrakt i sadržaj, i to sve mora da stane na prvu stranu! Kako bi Vaš seminarski zadovoljio standarde i očekivanja, koristite uputstva i materijale sa predavanja na temu pisanja seminarskih radova. Ovo je samo šablon koji se odnosi na fizički izgled seminarskog rada (šablon koji \emph{morate} da ispoštujete!) kao i par tehničkih pomoćnih uputstava. Molim Vas da kada budete predavali seminarski rad, imenujete datoteke tako da sadrže temu seminarskog rada, kao i imena i prezimena članova grupe (ili samo temu i prezimena, ukoliko je sa imenima predugačko). Predaja seminarskih radova biće isključivo preko web forme, a NE slanjem mejla.

\tableofcontents

\newpage

\section{Uvod}
\label{sec:uvod}

U poslednjih 30 godina svedoci smo neverovatnog napretka kompjuterskih tehnologija. 
Računari su za veoma kratko vreme prešli put od čuda tehnike koje je dozvoljeno posmatrati isključivo kroz stakleni zid serverske sobe, 
do uređaja čije se korišćenje podrazumeva u svakodnevnom radu čoveka. Broj uređaja koje možemo smatrati računarima se konstantno povećava iz godine u godinu. 
Od mobilnih telefona, preko prenosnih računara, pa sve do pametnih veš-mašina, računari su postali sastavni (vrlo često i centralni) deo uređaja za koje, 
do pre samo par godina, nismo mogli ni da pretpostavimo da će postati "pametni".
\par
Ovaj fenomen eksplozije broja računara i njegovih primena deoveo je do nekih logičnih posledica. 
Arhitektura procesora od uređaja do uređaja postala je izuzetno šarolika. Zahtevi za uštedom memorije, procesorskog vremena i energije (struje) postali su primarni na uređajima koji nisu 
(i ne mogu biti) na konstantnom izvoru napajanja. Optimizacija je ponovo postala ključni deo problema u razvoju softvera. 

\par
Sa druge strane postoji velika potražnja za softverom koji rešava raznovrsne problema, a koji se pritom izvršava na još raznorodnijim arhitekturama i platformama.
Veliki broj problema koji može biti rešen softverski sa jedne strane i nedovoljan broj razvijaoca softvera sa druge nam govori da je brz razvoj sofvera neophodan kako bi svi problemi 
mogli da se reše u razumnom roku. Način na koji industrija odgovara na taj izazov godinama unazad bio je uopštavanje arhitekture računara koji je omogućio razvijaocima softvera 
da na gotovo isti (ili veoma sličan) način i sa istim alatima i jezicima razvijaju softver za veoma različite platforme. Posledica takvog pristupa je da uopšteni model arhitekture gotovo uvek zanemaruje 
naprednije mogućnosti modernih arhitektura procesora. Korišćenjem samo osnovnih hardverskih funkcionalnosti procesora dolazimo do programa koju su veoma malo (ako uopšte) optimizovani.

\par 
Moderni programski prevodioci uglavnom primenjuju neke statičke tehnike optimizacije, primenjujući različite \emph{transformacije} koda koje čuvaju semantičko značenje, kako bi proizveli što optimizovaniji izvorni kod, ali tu nastaju novi problemi. 
Niz transformacija koji daje dobre rezultate na jednoj arhitekturi i za jedan tip algoritamskog problema, na drugoj arhitekturi i za drugi tip problema može proizvesti rezultat koji je lošiji od neoptimizovanog rešenja.
Rešenje ovog problema se nazire u konfigurabilnosti redosleda i ulaznih parametara ovih transformacija.
\par
U ovom radu opisujemo jedan pristup optimizaciji u procesu prevođenja koji je nezavisan od platforme i arhitekture. Suština ove metode prevođenja je u dodatnom sloju koji se nalazi između izvornog koda i prevodioca. Svrha tog sloja je da proizvede optimizovaniji \emph{izvorni k\^{o}d} koji se zatim prosleđuje 
prevodiocu. Kako postoje različiti načini transformacije jednog izvornog koda u drugi, i kako ne postoji jedan statički niz transformacija i njihovih konfiguracija koji bi dao optimalne rezultate za sve probleme i platforme, 
jasno je da odabir transformacija i ulazne parametre tih tranformacija moramo odabirati naći nekakvom pametnom pretragom. Taj proces pretrage parametara nazivamo \emph{"Iterativna kompilacija"}.
\par
Ovaj rad organizovan je na sledeći način. 
U odeljku \ref{sec:tehnike} opisujemo neke od transformacija izvornog koda koje je moguće koristiti pri optimizaciji. 
Koncentrišemo se (naravno) na one transformacije koje su konfigurabilne. Pokazujemo na koji način ove transformacije utiču na performanse i kako se mogu kombinovati. 
U odeljku \ref{sec:tehnike} poredimo različite metode pretrage ulaznih parametara niza transformacija.
U odeljku \ref{sec:performanse} poredimo performanse programa prevedenih na ovaj način kroz različite repere i realne programe, i poredimo potrošnju energije tih programa.
Konačno, dajemo zaključak na osnovu rezultata merenja performansi i potrošnje energije u odeljku \ref{sec:zakljucak}.


\section{Tehnike optimizacije}
\label{sec:tehnike}

Slike i tabele treba da budu u svom okruženju, sa odgovarajućim naslovima, obeležene labelom da koje omogućava referenciranje. 

\begin{primer} Ovako se ubacuje slika. Obratiti pažnju da je dodato i 
\begin{verbatim}
\usepackage{graphicx}
\end{verbatim}

\begin{figure}[h!]
\begin{center}
% [TODO: staviti placeholder za sliku] \includegraphics[scale=0.75]{panda.jpg}
\end{center}
\caption{Pande}
\label{fig:pande}
\end{figure}

Na svaku sliku neophodno je referisati se negde u tekstu. Na primer, na slici \ref{fig:pande} prikazane su pande. 
\end{primer}

\begin{primer} I tabele treba da budu u svom okruženju, i na njih je neophodno referisati se u tekstu. Na primer, u tabeli \ref{tab:tabela1} su prikazana različita poravnanja u tabelama.

\begin{table}[h!]
\begin{center}
\caption{Razlčita poravnanja u okviru iste tabele ne treba koristiti jer su nepregledna.}
\begin{tabular}{|c|l|r|} \hline
centralno poravnanje& levo poravnanje& desno poravnanje\\ \hline
a &b&c\\ \hline
d &e&f\\ \hline
\end{tabular}
\label{tab:tabela1}
\end{center}
\end{table}

\end{primer}

\subsection{Tehnike optimizacije petlji}
\subsubsection{Odmotavanje petlji}
\subsubsection{Popločavanje petlji}
\subsubsection{Neka treća optimizacija}

\subsection{Odabir ulaznih parametara}



\section{Performanse}
\label{sec:performanse}


Ovde pišem tekst. 
Ovde pišem tekst. 
Ovde pišem tekst. 
Ovde pišem tekst. 
Ovde pišem tekst. 
Ovde pišem tekst. 
Ovde pišem tekst. 
Ovde pišem tekst. 


\subsection{Statičke metode}
\label{subsec:podnaslov1}

Ovde pišem tekst. 
Ovde pišem tekst. 
Ovde pišem tekst. 
Ovde pišem tekst. 
Ovde pišem tekst. 
Ovde pišem tekst. 
Ovde pišem tekst. 

\subsection{Metode testiranja iterativne kompilacije}
\label{subsec:podnaslov2}

Ovde pišem tekst. 
Ovde pišem tekst. 
Ovde pišem tekst. 
Ovde pišem tekst. 
Ovde pišem tekst. 
Ovde pišem tekst. 

\subsection{Rezultati izvršavanja nad reperima}
\label{subsec:podnaslov3}

Ovde pišem tekst. 
Ovde pišem tekst. 
Ovde pišem tekst. 
Ovde pišem tekst. 

\section{Zaključak}
\label{sec:zakljucak}

Ovde pišem zaključak. 
Ovde pišem zaključak. 
Ovde pišem zaključak. 
Ovde pišem zaključak. 
Ovde pišem zaključak. 
Ovde pišem zaključak. 
Ovde pišem zaključak. 
Ovde pišem zaključak. 
Ovde pišem zaključak. 
Ovde pišem zaključak. 
Ovde pišem zaključak. 
Ovde pišem zaključak. 


%\addcontentsline{toc}{section}{Literatura}
%\appendix
%\bibliography{seminarski} 
%\bibliographystyle{plain}

\appendix
\section{Dodatak}
Ovde pišem dodatne stvari, ukoliko za time ima potrebe.
Ovde pišem dodatne stvari, ukoliko za time ima potrebe.
Ovde pišem dodatne stvari, ukoliko za time ima potrebe.
Ovde pišem dodatne stvari, ukoliko za time ima potrebe.
Ovde pišem dodatne stvari, ukoliko za time ima potrebe.


\end{document}
